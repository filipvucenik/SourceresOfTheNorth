\chapter{Zaključak i budući rad}
		
		Projektni zadatak našeg tima bio je razvoj web aplikacije koja omogućuje građanima da prijavljuju razna oštećenja gradskim uredima. S druge strane aplikacija omogućuje djelatnicima gradskih ureda da upravljaju s pristiglim prijavama.
		
		Nakon gotovo 4 mjeseca predanog i kontinuiranog rada možemo reći da smo ostvarili ciljeve projekta te putem stekli vrijedna iskustva i usvojili brojne vještine. Proces razvoja projekta je bio logički podijeljen u dvije faze.
		
		Prva faza je započela formiranjem tima i upoznavanjem sa projektnim zadatkom. Nakon toga je slijedio period u kojem smo raspravljali o zadatku, razmjenjivali ideje o oblikovanju programske potpore i izrađivali skice za osnovni dizajn korisničkog sučelja. Iz svega toga smo izlučili, popisali i dokumentirali funkcijske i nefunkcijske zahtjeve. Na temelju funkcijskih zahtjeva izrađeni su obrasci uporabe te dijagrami obrasca uporabe i sekvencijski dijagrami. Također je raspravljen i razrađen dizajn baze podataka sa popratnim relacijskim dijagramom. Paralelno s tim članovi tima su podijeljeni na tim za razvoj frontend dijela aplikacije i tim za razvoj backend dijela aplikacije. Za oba dijela aplikacije su odabrane tehnologije pomoću kojih će se ti dijelovi razvijati. S obzirom da su odabrane tehnologije bile nove za članove tima, u ovoj fazi je dio vremena trebao biti odvojen za upoznavanje i usvajanje tih tehnologija. Krajem prve faze definirana je arhitektura sustava na temelju odabranih tehnologija te je na temelju toga napravljen dijagram razreda. Članovi backend i frontend tima su započeli implementirati sustav na temelju dogovorenih zahtjeva. Na kraju prve faze realizirane su osnovne funkcionalnosti sustava, kao što su prijava, registracija i osnovni obrazac za slanje prijava, i napravljen osnovni dizajn korisničkog sučelja.
		
		U drugoj fazi je glavni fokus bio na ostvarivanju svih funkcionalnosti koje smo definirali i dokumentirali u prvoj fazi projekta. Paralelno s time u dokumentaciji je bilo potrebno izraditi nekoliko UML dijagrama koji opisuju sustav iz više perspektiva, popisati sve korištene tehnologije i alate tijekom trajanja projekta te napisati upute za puštanje aplikacije u pogon. Na kraju je odrađeno i dokumentirano ispitivanje komponenti i sustava u cjelini.
		
		Prva faza projekta je sadržavala više aktivnosti te je tekla dosta sporije. Glavni razlog tome je bilo potrebno vrijeme da se članovi tima uhodaju i upoznaju s projektom. Također je značajan faktor bio što nitko od članova tima nije imao iskustva sa timskim radom na projektu ovakvog tipa niti iskustva sa glavnim tehnologijama koje su se koristile tijekom projekta. Stoga je bilo potrebno vrijeme da se te prepreke savladaju.
		
		Druga faza projekta nije sadržavala toliko aktivnosti, ali je bila izuzetno izazovna jer je bilo potrebno ostvariti veći dio funkcionalnosti u relativno kratkom vremenu.
		
		Tim je imao redovite sastanke gotovo svaki tjedan na kojima smo raspravljali o onome što smo napravili u proteklom tjednu te raspravljali na čemu bi trebali raditi dalje. Ovime se maksimalno poticalo da svi članovi tima doprinose projektu na tjednoj bazi, nastojala se poboljšati komunikacija između članova tima te ažurnost svakog člana. Kroz projekt se brzo moglo vidjeti da su jedni od bitnih elemenata dobro odrađenog projekta dobra organizacija i koordinacija tima, redovit i kontinuiran rad, otvorena komunikacija te fleksibilnost članova tima da reagiraju na probleme koji su se javljali tijekom projekta. Jedna od glavnih prepreka je bila nedostatak prijašnjeg iskustva svih članova tima na ovakvim projektima. Također, nepoznavanje tehnologija koje smo kao tim koristili kroz projekt se također pokazalo kao jedna od većih prepreka. To je dosta usporavalo razvoj projekta zbog duljeg vremena potrebnog za uhodavanje i upoznavanje tehnologija i timskog načina rada, pogotovo u prvoj fazi projekta.
		
		Na kraju projekta možemo reći da smo zadovoljni postignutim i naučenim. Sama aplikacija ima puno prostora za unaprjeđenje, ali kroz ovaj projekt smo dobili prijeko potrebno iskustvo timskog rada na projektnim zadacima. Također smo usvojili nova znanja i savladali korištenje novih tehnologija što će nam biti izuzetno korisno u budućem radu. 
		\eject 