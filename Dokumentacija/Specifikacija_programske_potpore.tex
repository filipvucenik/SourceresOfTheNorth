\chapter{Specifikacija programske potpore}
		
	\section{Funkcionalni zahtjevi}
			
			\textbf{\textit{dio 1. revizije}}\\
			
			\noindent \textbf{Dionici:}
			
			\begin{packed_enum}
				\item Gradska uprava
				\item Djelatnici gradskih ureda		
				\item Građani grada
				\item[] \begin{packed_item}
					\item Registrirani korisnici
					\item Neregistrirani korisnici
				\end{packed_item}
				\item Razvojni tim
			\end{packed_enum}
			
			
			\noindent \textbf{Aktori i njihovi funkcionalni zahtjevi:}
			
			\begin{packed_enum}
				
				\item  \underbar{Neregistrirani korisnik (inicijator) može:}
				\begin{packed_enum}
					\item napraviti svoj profil za koji mu je potrebno ime, prezime, datum rođenja, email adresa, lozinka i opcionalno profilna slika
					\item podnositi anonimnu prijavu koja sadrži naslov, opis, lokaciju i nadležni gradski ured, a opcionalno i fotografiju i prijavu na koju se nadovezuje
					\item  pregledavati postojeće prijave preko liste ili karte te ih filtrirati po lokaciji, vremenu, statusu i nadležnom gradskom uredu
					\item pregledavati statistiku postojećih prijava za odabran period, lokaciju i nadležan gradski ured			
				\end{packed_enum}
								
				\item  \underbar{Registrirani korisnik (inicijator) može:}				
				\begin{packed_enum}					
					\item pregledavati i mijenjati osobne podatke
					\item obrisati svoj profil
					\item pregledavati svoje prošle prijave
					\item podnositi prijavu koja sadrži naslov, opis, lokaciju i nadležni gradski ured, a opcionalno i fotografiju i prijavu na koju se nadovezuje
					\item  pregledavati postojeće prijave preko liste ili karte te ih filtrirati po lokaciji, vremenu, statusu i nadležnom gradskom uredu
					\item pregledavati statistiku postojećih prijava za odabran period, lokaciju i nadležan gradski ured
				\end{packed_enum}
				
				\item  \underbar{Djelatnik gradskog ureda (inicijator) može:}
				\begin{packed_enum}
					\item pregledavati prijave pristigle u njihov ured prema njihovom statusu
					\item prijavama sa statusom \textit{na čekanju} mijenjati status u \textit{u procesu rješavanja}
					\item prijavama sa statusom \textit{u procesu rješavanja} mijenjati status u \textit{riješena}
					\item objediniti prijave ako se referiraju na isti problem
					\item pregledavati prijave na karti
					\item pregledavati statistiku prijava pristiglih u njihov ured
				\end{packed_enum}
				
				\item  \underbar{Baza podataka (sudionik) može:}
				\begin{packed_enum}
					\item spremati sve podatke o korisnicima
					\item spremati sve podatke o prijavama
				\end{packed_enum}
				
			\end{packed_enum}
						
			\eject 
			
				
						
			\subsection{Obrasci uporabe}
				
				\textbf{\textit{dio 1. revizije}}
				
				\subsubsection{Opis obrazaca uporabe}
					\textit{Funkcionalne zahtjeve razraditi u obliku obrazaca uporabe. Svaki obrazac je potrebno razraditi prema donjem predlošku. Ukoliko u nekom koraku može doći do odstupanja, potrebno je to odstupanje opisati i po mogućnosti ponuditi rješenje kojim bi se tijek obrasca vratio na osnovni tijek.}\\
					

					\noindent \underbar{\textbf{UC1 -Prijava oštećenja}}
					\begin{packed_item}
	
						\item \textbf{Glavni sudionik:} Korisnik
						\item  \textbf{Cilj:} Prijaviti oštećenje nadležnom gradskom uredu
						\item  \textbf{Sudionici:} Baza podataka
						\item  \textbf{Preduvjet:} Korisnik je pronašao oštećenje koje želi prijaviti
						
						\item  \textbf{Opis osnovnog tijeka:}
						\item[] \begin{packed_enum}
							\item Korisnik odabire opciju za prijavljivanje oštećenja
							\item Korisnik popunjava podatke za prijavu oštećenja i podnosi prijavu
							\item Prijava se sprema u bazu podataka
							\item Korisnik dobiva jedinstveni kod svoje prijave.
						\end{packed_enum}
						
						\item  \textbf{Opis mogućih odstupanja:}
						\item[] \begin{packed_item}
							\item[2.a] Korisnik nije unio sve obavezne podatke u prijavu.
							\item[] \begin{packed_enum}
								\item Korisnik dobiva obavijest.
								\item Korisnik popunjava preostala obavezna polja ili odustaje od podnošenja prijave
							\end{packed_enum}
							
						\end{packed_item}
					\end{packed_item}
					
					
					\noindent \underbar{\textbf{UC2 -Pregled svih prijava}}
					\begin{packed_item}
						
						\item \textbf{Glavni sudionik:} Korisnik
						\item  \textbf{Cilj:} Pregled aktivnih prijava u sustavu
						\item  \textbf{Sudionici:} Baza podataka
						\item  \textbf{Preduvjet:} -
						
						\item  \textbf{Opis osnovnog tijeka:}
						\item[] \begin{packed_enum}
							\item Korisnik odabire opciju za pregledavanje prijava
							\item Aplikacija prikazuje prijave
							\item Korisnik odabire prijavu
							\item Prikazuju se podaci o prijavi
						\end{packed_enum}
					\end{packed_item}
				
				
				\noindent \underbar{\textbf{UC3 -Filtriranje prijava}}
				\begin{packed_item}
					
					\item \textbf{Glavni sudionik:} Korisnik
					\item  \textbf{Cilj:} Filtriranje prijava prilikom pregleda
					\item  \textbf{Sudionici:} Baza podataka
					\item  \textbf{Preduvjet:} Korisnik je odabrao opciju pregleda svih prijava
					
					\item  \textbf{Opis osnovnog tijeka:}
					\item[] \begin{packed_enum}
						\item Korisnik odabire opcije kod filtriranja
						\item Aplikacija prikazuje filtrirane prijave
					\end{packed_enum}
					
					\item  \textbf{Opis mogućih odstupanja:}
					\item[] \begin{packed_item}
						\item[1.a] Korisnik ne želi filtrirati prijave
						\item[] \begin{packed_enum}
							\item Aplikacija prikazuje nefiltrirane prijave
						\end{packed_enum}
						
					\end{packed_item}
				\end{packed_item}
				
				
				\noindent \underbar{\textbf{UC4 -Pregled statistike prijava}}
				\begin{packed_item}
					
					\item \textbf{Glavni sudionik:} Korisnik
					\item  \textbf{Cilj:} Pregled statistike prijava u sustavu
					\item  \textbf{Sudionici:} Baza podataka
					\item  \textbf{Preduvjet:} -
					
					\item  \textbf{Opis osnovnog tijeka:}
					\item[] \begin{packed_enum}
						\item Korisnik odabire opciju za pregledavanje statistike prijava
						\item Aplikacija prikazuje razne podatke za prijave u sustavu
					\end{packed_enum}
				\end{packed_item}
				
				
				\noindent \underbar{\textbf{UC5 -Filtriranje statistike prijava}}
				\begin{packed_item}
					
					\item \textbf{Glavni sudionik:} Korisnik
					\item  \textbf{Cilj:} Filtriranje prijava prilikom pregleda statistike
					\item  \textbf{Sudionici:} Baza podataka
					\item  \textbf{Preduvjet:} Korisnik je odabrao opciju pregleda statistike prijava
					
					\item  \textbf{Opis osnovnog tijeka:}
					\item[] \begin{packed_enum}
						\item Korisnik odabire opcije kod filtriranja
						\item Aplikacija prikazuje statistiku za filtrirane prijave
					\end{packed_enum}
				\end{packed_item}
					
				\subsubsection{Dijagrami obrazaca uporabe}
					
					\textit{Prikazati odnos aktora i obrazaca uporabe odgovarajućim UML dijagramom. Nije nužno nacrtati sve na jednom dijagramu. Modelirati po razinama apstrakcije i skupovima srodnih funkcionalnosti.}
				\eject		
				
			\subsection{Sekvencijski dijagrami}
				
				\textbf{\textit{dio 1. revizije}}\\
				
				\textit{Nacrtati sekvencijske dijagrame koji modeliraju najvažnije dijelove sustava (max. 4 dijagrama). Ukoliko postoji nedoumica oko odabira, razjasniti s asistentom. Uz svaki dijagram napisati detaljni opis dijagrama.}
				\eject
	
		\section{Ostali zahtjevi}
		
			\textbf{\textit{dio 1. revizije}}\\
		 
			 \textit{Nefunkcionalni zahtjevi i zahtjevi domene primjene dopunjuju funkcionalne zahtjeve. Oni opisuju \textbf{kako se sustav treba ponašati} i koja \textbf{ograničenja} treba poštivati (performanse, korisničko iskustvo, pouzdanost, standardi kvalitete, sigurnost...). Primjeri takvih zahtjeva u Vašem projektu mogu biti: podržani jezici korisničkog sučelja, vrijeme odziva, najveći mogući podržani broj korisnika, podržane web/mobilne platforme, razina zaštite (protokoli komunikacije, kriptiranje...)... Svaki takav zahtjev potrebno je navesti u jednoj ili dvije rečenice.}
			 
			 
			 
	